\chapter*{General Problem statement}
\label{section:generalproblemstatment}
\addcontentsline{toc}{chapter}{General Problem statement}
Today, mammography is the primary imaging modality for breast cancer screening and plays an important role in cancer diagnosis. Subtle soft-tissue findings and microcalcifications that may represent early breast cancer are visualized by means of X-rays images. After investigation, the abnormal findings are taken in charge for further evaluation.  
A standard mammographic protocol always includes breast compression prior to image acquisition. Women breast is compressed between two plates until a nearly uniform breast thickness is obtained. The breast flattening improves diagnostic image quality1 and reduces the absorbed dose of ionizing photons2. However, the discomfort and pain produced by this procedure sometimes might deter women from attending breast screening by mammography 3. Fleming et al4 show in a study of 2500 women that 15\% of those who skipped the second appointment cited an unpleasant or painful first mammogram. 
Nowadays, the European Commission recommends a force standardized breast compression, i.e. the compression stops at a level of force just below the subject’s pain threshold or to the maximum setting of the machine (not to exceed 200 N). Some research5 indicates that with a reduced level of compression (10N vs 30N), 24\% of women did not experience a difference in breast thickness. If breast thickness is not reduced when compression force is applied, then discomfort is increased with no benefit in image quality. 
An important improvement concerning the patient comfort could be achieved with the emergence of Full-Field Digital Mammography (FFDM).  Several studies have shown that digital mammography is better in terms of image quality 6,7 and radiation dose 2,8 than Film–Screen Mammography (FSM). Therefore, there is an opportunity to leverage the potential of the recent imaging technologies to investigate alternative breast compression techniques, considering the patient comfort in addition to an improved image quality and a reduced ionizing radiation dose. The aim of this work is to develop and evaluate a biomechanical Finite Element (FE) breast model allowing to investigate alternative breast compression strategies. 




\chapter*{Technical approach}\label{section:technicalapproach}
\addcontentsline{toc}{chapter}{Technical approach}
Several studies showed that, the pain experienced by women during the mammographic exam depends on psychologic factor \citep{aro_pain_1996} (technician behavior, patient anxiety), sociologic factors \citep{dullum_rates_2000} (ethnicity, education level) as well as physiologic \citep{poulos_breast_2003} factors (compression level,  breast size). Here, the psychologic and sociologic factors are neglected. The study focuses on physiological factors as the compression force or structural specifications of the compression paddle to characterize the patient comfort. 


\chapter*{Thesis overview}\label{section:thesisoverview}
\addcontentsline{toc}{chapter}{Thesis overview}
\chapter*{Key contributions}\label{section:keycontributions}
\addcontentsline{toc}{chapter}{Key contributions}
\chapter*{Software}\label{section:software}
\addcontentsline{toc}{chapter}{Software}
\chapter*{Ethics}\label{section:ethics}
\addcontentsline{toc}{chapter}{Ethics}