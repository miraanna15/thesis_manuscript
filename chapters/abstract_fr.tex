\noindent
\textbf{Contexte} La mammographie est une modalité d’imagerie médicale à faible dose d'irradiation permettant la détection du cancer à une stade précoce. Pendant l'examen, le sein est comprimé entre deux plaques afin d'uniformiser l'épaisseur du sein et d'étaler les tissus mous.  Cette technique amélior la qualité de l'examen mais est aussi considéré une potentiel source d'inconfort et parfois de douleur pour la patiente. Bien que la mammographie soit la méthode de dépistage du cancer du sein la plus efficace, l’inconfort ressentie pendant l'examen peut dissuader les femmes de passer cet examine. Par conséquent, une technique alternative de compression du sein permettant en compte le confort du patient en plus de l’amélioration de la qualité d'image présente un grand intérêt.

\noindent
\textbf{Méthodes}

\noindent
\textbf{Résultats}

\noindent
\textbf{Conclusion}

\noindent
\textbf{Mots clé}