
\chapter{Conclusion and Perspectives}\label{section:generalconclusion}
\addcontentsline{toc}{chapter}{Conclusion}


The main aim of this PhD work was to develop a simulation framework capable of assessing breast compression as a function of paddle design, compression force or breast positioning. Image quality, average glandular dose as well as patient comfort were considered when comparing different compression strategies.

 A simulation software was used to compute the X-ray propagation through matter and to generate simulated mammography images. The image quality was depending on the compressed breast thickness. The average glandular dose was computed using the method proposed by Dance et al. \citep{dance_additional_2000}. 
 
 To assess the breast deformations depending on the paddle design or the applied compression force, a new biomechanical breast model was developed. After being evaluated by confrontation with real data collected through MR images, this model was used to predict the outer shape of the breast under compression as well as	 the patient comfort assumed to be associated to internal strain and stress intensities and distributions.\\
 
 
 
 In this chapter, the main results and conclusions on the implemented application are recalled. The possible improvements for a large prospect of applications are discussed. 
 \clearpage
 \section{Biomechanical breast model }
  Before modeling the mammography breast compression, the model fidelity to real tissues deformations had to be assessed. In this  scope, data sets describing in-vivo breast mechanics were collected. The MRI is an interesting imaging modality allowing the extraction of the whole 3D breast geometry with the distribution of each internal structure.  Therefore, MR images of two volunteers in three body positions (prone, supine and supine tilted) were acquired and used in this study. The two volunteers were chosen such that two different breast sizes were represented: small volume (breast cup A) and large volume (breast cup F). The boundary conditions describing breast deformation under gravity loading are easy to reproduce in a simulation framework. Thus, the acquired data made possible the calibration and the evaluation of our biomechanical model.
   
The development of the biomechanical breast model was performed using the small breast volume (first volunteer). A small breast volume undergoes smaller deformations under gravity loading, thus the corresponding breast model was suspected to lead to less convergence difficulties. To this end, the MRI breast volume of the first volunteer in supine position was used to build the finite element mesh. Then, combined with the MRI breast volume in prone body position, it was used to estimate the subject-specific constitutive parameters and the corresponding stress-free geometry. 
 
 From the first performed simulations, the role of sliding boundary conditions at the juncture surface between  the pectoral muscle and the breast was pointed out. Indeed, the displacements driven by only tissues' elasticity were not enough to reflect the geometrical changes between the supine and prone breast configurations. Therefore, breast tissues were allowed to slide over the pectoral muscle surface. Additional boundary conditions were considered by modeling the breast suspensory ligaments and fascial system. Including stiffer structures into the finite element model improved solution convergence capabilities. They also allowed to compute the breast deformations for a larger range of soft tissue constitutive parameters.
 
 According to the literature, a well defined breast model has to  consider in-vivo measured constitutive parameters. To this end, the tissues Young's moduli giving the best fit between the simulated and measured breast configurations of the first volunteer were computed. The optimal estimates, assuming Neo-Hookean material models, were found to be $\lambda_{breast}^r=0.3\ kPa$, $\lambda_{breast}^l=0.2\ kPa$, $\lambda_{skin}=4\ kPa$ and $\lambda_{fascia}=120\ kPa$. The obtained mechanical properties are comparable to the ones proposed in the literature when considering only the breast models with similar simulation frameworks  \citep{rajagopal_modelling_2007, gamage_modelling_2012, griesenauer_breast_2017}. This result allowed to compute the breast geometry in supine and prone configurations with an accuracy of $1.70\ mm$ and $2.17\ mm$ respectively. The model fidelity to the global breast deformation was evaluated using the breast geometry in supine tilted position. The Hausdorff distance between the breast skin surface extracted from the MRI volume and the simulated skin surface was equal to $6.14 \ mm$. The larger error ($\sim 26.03 \ mm$) was obtained on the left breast where the lateral displacements of the tissues were overestimated due to large strains observed over the fascia and skin surfaces. 
 
 These results may be improved by developing a more complex breast support matrix or considering region dependent stiffness for the skin or the fascia system. Adding stiffer materials in the strategically localized areas on the contact surface may limit the fascia' deformations under large strain range. Consequently, the breast tissues sliding in supine tilted configuration may be reduced.
 
When developing a subject-specific finite element model, the source of error may be both an inaccurate definition of model components or an imprecise model calibration. One may observe that, the objective function used during the model optimization does not properly consider the tissues lateral displacements. The supine and prone breast configurations are not sufficient to describe the soft tissue sliding over the pectoral muscle. Therefore, new information characterizing such breast deformations has to be added. We believe that, considering the supine tilted breast configuration, as a target volume, when performing the optimization of the constitutive parameters,  may result in a more accurate estimation of tissues' material models. However, in such case, additional volumes with different breast configurations would be needed to perform the model evaluation. Furthermore, in this work, the similarity between the estimated and measured breast configurations was assessed by computing the minimal distance from the estimated position of each skin node to the measured breast shape.  Using only surface measurements does not entirely describe how both the skin surface and the internal tissues deform. During the MRI acquisitions, both volunteers were asked to fix six fiducial landmarks over the breast surface, in order to enable volumes registration. Even if this is difficult, one could envisage to identify internal anatomical landmarks defined by singular structures within breast volume (i.e. clusters of microcalcifications), in order to extend the registration process to internal structures of the breast. Such additional landmarks may bring new information about the internal tissue deformations. In a future work, considering this type of information as additional constraints in the calibration process may provide a more realistic estimation of breast mechanical properties. Moreover, including the distance between landmarks location in addition to the distance between breast skin surfaces should also provide a more accurate model evaluation. 

After performing the calibration process using the data of the smaller breast volume (first volunteer), the breast biomechanical model was built up for the larger breast volume (second volunteer). However, because of a lack of time,  the model calibration process was not performed for this volunteer. In future work, it would be interesting to perform such model optimization and evaluation on at least two more subjects with different breast morphologies and mechanical properties. The model calibration on a larger population should improve its flexibility for further studies on breast compression techniques.

Despite providing good results in the multi-loading gravity framework, the developed breast model turned out to be less efficient in simulating breast compression. With this model, the maximal force needed to simulate breast flattening was estimated to be relatively low when compared to the mean recommended force of a typical mammography exam ($10 N$ vs $120 N$). These low values of the compression force were due to tissue's abnormal softening under large stress rates. Tissue relaxation from a given stress threshold is a well known phenomenon when using Neo-Hookean materials. We therefore suggested to overcome this issue by replacing the Neo-Hookean model by a Gent model for all involved hyperelastic tissues.

The stress-strain relation remains almost the same for both models below a strain threshold defined by the parameter  $J_m$. Beyond this threshold the Gent material model is stiffening exponentially resulting in an asymptotic behavior. These properties allow to change the tissues mechanical response only for large strain rates, as during breast compression. On the other hand, they also allow to preserve the same mechanical response for relatively small strains as induced by gravity loading simulations.

The breast compression simulations were performed using both breast volumes. As the tissues constitutive parameters of the second volunteer were not optimized, the values proposed by others similar works were used. Therefore, the second breast model does not describe the subject-specific mechanical behavior. However, it provides  a new realistic model that may be used to assess the compression quality of different compression types. The Gent material model improved the tissues mechanical response when the breast is compressed between the paddle and the image receiver. A compression force of $22 \ N$ for the first volunteer and $95 \ N$ for the second volunteer were obtained to reach the target breast thicknesses, which fits well the measured clinical data ($21.9 \ N$ and $94.8 \ N$ respectively).  

Based on the same constitutive parameters, as estimated during breast compression, the Gent tissues model was used to estimate breast deformation under gravity loading for the first volunteer. We found that, with such a model, tissues displacements were over-constrained in both supine and prone configurations. However, we proved that the Gent model may also improve the biomechanical model fidelity to the real deformations under gravity loading. Therefore we suggested to use two different values of the parameter $J_m$ in order to obtain the best estimates for multi-loading gravity and breast compression simulations. A more detailed study has then to be considered in order to estimate a unique tissues constitutive model. To this end, more data describing the compression breast process, such as breast position with respect to paddle, is needed. The impact of using different constitutive parameters within tissues types should also be studied. However, increasing the number of model parameters might increase the complexity of the optimization problem, as well as the computation time and the size of the data set describing the subject breast mechanics.

In this work, we have assumed that the breast mechanics does not depend on the glandular tissues distribution. This assumption is only valid if a global breast deformation is assessed, such as the breast surface deformation. However, when one wants to estimate the breast internal tissues displacements under compression or gravity loading, a more subtle definition of tissues materials is needed. In such a case, the differentiation of the glandular and adipose tissues should be more relevant.   
   
During the modeling process, it has been observed that the breast tissues undergo extremely large deformations (up to 150\%). This involves large convergence difficulties due to bad element formulation during simulations. To improve solution convergence capabilities, an hexahedral adaptive mesh might be considered.

In conclusion, the proposed biomechanical model was able to provide a good estimation of breast deformation under different boundary conditions. The models based on breast geometries of two subjects were used to perform comparative studies between various compression methods.
 
\section{Breast compression and patient comfort}
The developed biomechanical breast model together with the image simulation framework were used to assess the clinical compression quality whithin different compression strategies.

A first study was performed to compare the breast compression quality when using standard rigid or flex paddles. To compute breast compressions, the proposed biomechanical model was calibrated for two  breast volumes (small and large) with various mechanical properties (soft and firm). The results showed that, using the flex paddle may improve the patient comfort without affecting the image quality and the delivered average glandular dose. Moreover, despite a breast thickness varying linearly from chest wall to nipple, the image quality seems to be preserved or even improved compared to the image quality obtained with a rigid compression paddle. Such a better image quality obtained when using the flex paddle could be explained by a better overall breast compression. The paddle tilt allows a better compression of the tissues close to the nipple, relaxing the tissues located in the chest wall region. Our simulations tend to confirm that the paddle tilt may displace breast tissues towards the chest wall. The tissues accumulation on the retromammary space may hide clinical relevant information and thus may increase the risk of false negatives. 

A second study was performed to assess the impact of breast positioning on breast compression mechanics. This time, the paddle deflection due to its material elastic properties was considered to enable solution convergence when the compression paddle is close to the chest wall. Three breast compression simulations were performed with various distances between the paddle and the chest wall. In this context, only the larger breast volume was used, the smaller volume being not adapted to such a simulation. The results showed that patient comfort can be improved by positioning the paddles further from the pectoral muscle. For an equivalent breast thickness under compression, the compression force decreased from $158\ N$ to $59\ N$ for a difference of $15\ mm$ in the distance from chest wall to paddle. On the other hand, clinical guidelines request to place the paddle as close as possible to the chest wall. Therefore, our simulations show that one may have to find a compromise between the exam quality and the patient comfort. 

These two preliminary studies have shown that the clinical compression quality can be assessed using such simulation frameworks. The tools designed during this PhD thesis may be used to perform wider studies on existing breast compression systems, but also to provide a first estimation of the performance of a new, not yet implemented, paddle design. Simulation based studies are less expensive in time and materials than the usual clinical studies, therefore they may be used to discharge the less relevant paddle models.

\cleardoublepage



\chapter*{Key contributions}\label{section:keycontributions}
\addcontentsline{toc}{chapter}{Key contributions}
\markboth{\textsc{Key contributions}}{}

The key contributions concerning the finite element breast modeling are the following:

\begin{itemize}
\item We introduced new boundary conditions which reflect the motion of the breast over the thoracic cage.  They included sliding contact surface based on Coulombs friction law combined with stiff support structures. To our knowledge,  the breast support matrix was considered for the first time into the finite element model. The generic model includes suspensory ligaments together with the fascial system, and was built based on their anatomical description.   

\item The iterative algorithms allowing to estimate the breast stress-free  geometry are generally using only one breast configuration (supine or prone breast configuration). In this work we proposed an optimization algorithm which estimate the breast stress-free geometry starting from supine configuration and then iteratively correct it based on the prone configuration. 

\item  We disposed of a data set of breast MR images in three different body positions of the same volunteer. Therefore, our biomechanical breast model was first calibrated using prone and supine configurations. Then, its mechanical response was evaluated on a third breast configuration (supine tilted). Because of a lack of reliable data, none of previously published biomechanical breast models was evaluated in such wide panel of deformations.  

\item We evaluated the capability of the proposed biomechanical breast model to reproduce the breast compression mechanics as described by several clinical studies.  This analysis allowed us to point out the limitations of the Neo-Hookean strain energy function when modeling such large deformations.  Accordingly, a new material constitutive model defined by the Gent strain energy function was proposed.


\item We developed a simulation framework allowing to quantify breast compression in terms of image quality, average glandular dose and patient comfort. Due to its modularity, this framework supports different paddle designs and different breast geometries and compositions. 
\end{itemize}

\noindent
Using previously described tools, two studies assessing the breast compression quality were performed, demonstrating:

\begin{itemize}
\item The difference of the compression quality in terms of patient comfort, image quality and average glandular dose between a standard rigid and flex paddles was quantified. We have shown that, for the small breast, there was no difference in patient experience or image quality between the two paddles. However, using the flex paddle to compress the larger breast, may improve the patient experience without affecting the image quality or the average glandular dose.
\item The impact of breast positioning on the compression mechanics and patient comfort was analyzed. The obtained results have proved that the breast positioning have a significant impact on patient comfort. The surface pressure in the juxtathoracic area increased when the paddle was positioned closer to the chest wall.
\end{itemize}
A list of publications summarizing the results of this work is provided below.
\cleardoublepage
 
\chapter*{Publications}\label{section:publications}
\addcontentsline{toc}{chapter}{Publications}
\markboth{\textsc{Publications}}{}

\begin{description}

\item  Mîra, A., Payan, Y., Carton, A. K., de Carvalho, P. M., Li, Z., Devauges, V., \& Muller, S. (2018, March). Simulation of breast compression using a new biomechanical model. \textit {In Medical Imaging 2018:Physics of Medical Imaging} (Vol. 10573, p. 105735A). International Society for Optics and Photonics. \\

\item  Mîra A., Carton A.K., Muller S. \& Payan Y. (2018). Breast biomechanical modeling for compression optimization in digital breast tomosynthesis. \textit{Computer Methods in Biomechanics and Biomedical Engineering}, Lecture Notes in Bioengineering, A. Gefen and D. Weihs editors, pp. 29-35, DOI $10.1007/978-3-319-59764-5\_4$ \\

\item  Mîra A., Carton A.K., Muller S. \& Payan Y. (2016). Breast biomechanical modeling for compression optimization in digital breast tomosynthesis. \textit{Proceedings of the 22nd Congress of the European Society of Biomechanics (ESB2016)}
\end{description}

\cleardoublepage