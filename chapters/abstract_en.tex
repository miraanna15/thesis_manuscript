\noindent
\textbf{Background} Mammography is a specific type of breast imaging that uses low-dose X-rays to detect breast cancer at early stage. During the exam, the women breast is compressed between two plates in order to even out the breast thickness and to spread out the soft tissues. This compression improves the exam quality but can also be a source of discomfort. Though mammography is the most effective breast cancer screening method, the discomfort perceived during the exam might deter women from getting the test. Therefore, an alternative breast compression technique considering the patient comfort in addition to an improved clinical image quality is of large interest. \\

\noindent
\textbf{Methods}
In this work, a simulation environment allowing the evaluation of different breast compression techniques was put forward. The compression quality was characterized in terms of patient comfort, image quality and average glandular dose. To estimate the breast deformation under compression, a subject-specific finite element biomechanical model was developed. The model was calibrated and evaluated using MR images in three different breast configurations (supine, prone and supine tilted). On the other hand, image quality was assessed by using an already validated simulation framework. This framework was largely used to mimic image acquisitions in mammography. \\


\noindent
\textbf{Findings} The capability of our	 breast biomechanical model to reproduce the real breast deformations was evaluated. To this end, the geometry estimates of the three breast configurations were computed using Neo-Hookean material models. The subject specific mechanical properties of each breast's structures were assessed, to get the best estimates of the supine and prone configurations. The Hausdorff distances between the estimated and the measured geometries were equal to $2.17 \ mm$ and $1.72 \ mm$  respectively. Then, the model was evaluated using a supine tilted configuration with a Hausdorff distance of $6.14\ mm$. However, we have shown that the Neo-Hookean strain energy function cannot fully describe the rich mechanical behavior of breast soft tissues. Therefore, alternative material models based on the Gent strain energy function were proposed. The latter assumption reduced the maximal error in supine tilted breast configuration by about $10mm$. \\


\noindent
The coupling between the simulations of the breast mechanics and the X-ray simulations allowed us to run two preliminary studies. In the first study, the differences between standard rigid and flex compression paddles were assessed. According to the performed simulations, using the flex paddle for breast compression may improve the patient comfort without affecting the image quality and the delivered average glandular dose. 

\noindent
In the second study, the impact of breast positioning on the general compression mechanics was described. Our simulations confirm that positioning the paddle closer to the chest wall is suspected to increase the patient discomfort. Indeed, based on the estimated data, for the same breast thickness under compression, the force applied to the breast may increase by $150\%$. \\

\noindent
\textbf{Conclusion}
The good results we obtained for the estimation of breast deformation under gravity, as well as the conforming results on breast compression quality with the already published clinical statements, have shown the feasibility of such studies by the means of a simulation framework. \\

\noindent
\textbf{Keywords} Mammography, breast compression, patient comfort, biomechanical model 
