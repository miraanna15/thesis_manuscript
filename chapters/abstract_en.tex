\noindent
\textbf{Background} Mammography is a specific type of breast imaging that uses low-dose X-rays to detect cancer in early stage. During the exam, the women breast is compressed between two plates in order to even out the breast thickness and to spread out the soft tissues. This technique improves the exam quality but can be also a source of discomfort and sometimes pain. Though the mammography is the most effective breast cancer screening method, the discomfort perceived during the exam could deter women from getting the test. Therefore, an alternative breast compression technique considering the patient comfort in addition to an improved clinical image quality is of large interest. \\

\noindent
\textbf{Methods}
In this work, a simulation environment allowing to evaluate different breast compression techniques was proposed. The compression quality was characterized in terms of patient comfort, image contrast and average glandular dose. To assess the patient comfort, a subject-specific biomechanical model of the breast was developed providing physics-based predictions of the tissues mechanical response. The model was calibrated and evaluated using MR images in supine, prone and supine tilted body configurations. Then, it was used to estimate the breast deformation under compression.  The corresponding internal stress and strain intensity are assumed to be directly correlated with the patient discomfort. On the other hand, the image quality is assessed by using an already validated simulation framework. The latter is largely used to mimic image acquisitions in mammography. \\


\noindent
\textbf{Findings} Before using the breast biomechanical model to perform compression simulations, its capability to reproduce the real breast deformations was evaluated. To this end, the geometry estimates of the three breast configurations were computed using Neo-Hookean materials models with subject specific mechanical properties. Hausdorff distance between the estimated and the measured breast geometries for prone, supine and supine tilted configurations were equal to $ 2.17 mm$, $1.72 mm$ and $6.14 mm$ respectively. However, is was proved that the Neo-Hookean strain energy function cannot totally describe the rich mechanical behavior of breast soft tissues. Therefore, an alternative materials models based on the Gent strain energy function were proposed. The latter assumption improved the maximal error in supine tilted breast configuration by about $10mm$. \\

\noindent
The use of both technologies, the finite elements simulations and the X-ray simulations, allowed to perform two preliminary studies. In the first study, the differences between the standard rigid and flex compression paddles were assessed. According to the performed simulations, using the flex paddle for breast compression may improve the patient comfort without affecting the image quality and the delivered average glandular dose. However, the soft tissues are suspected to slide outside the projected breast area.

\noindent
In the second study, the impact of breast positioning on the compressed breast mechanics was described. Our simulations confirm that, positioning the paddle closer to the chest wall is suspected to increase the patient discomfort. According to the estimated data, for the same breast thickness, the compression force may be increased by $150\%$ \\

\noindent
\textbf{Conclusion}
The good estimation of breast deformation under gravity as well as the conforming results on breast compression quality with the already published clinical statements have shown the feasibility of such studies by the means of a simulation framework. \\

\noindent
\textbf{Key words} Mammography, breast compression, patient comfort, biomechanical model 
