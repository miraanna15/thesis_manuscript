
The state of the art in breast finite element modeling was presented in the previous chapter. Three models were identified representing the cutting-edge technologies in the field. The authors used prone MRI to estimate the breast reference geometry. The subject-specific constitutive parameters were chosen such that the best fit between the supine configuration estimate and the corresponding measured breast geometry was obtained. The models were evaluated using the measured and the estimated positions of superficial fiducial landmarks or internal anatomical landmarks in the supine breast configuration. As the optimization and evaluation  processes were based on the same data, the model limitations must have been underestimated. For a better assessment of the accuracy of a breast biomechanical model, we think that the evaluation on a third breast configuration is required.

 This chapter introduces a new biomechanical model developed by combining the best practices and concepts proved by previous works. To be as realistic as possible, our model considers breast heterogeneity, sliding boundary conditions, initial pre-stresses and personalized hyper-elastic properties of breast tissue. In addition, new types of soft tissue are included representing the breast support matrix. Moreover, our model was built using prone and supine breast configurations and was evaluated in supine tilted configuration (\~ 45 deg) of the same volunteer.\\
 

In the first part of this chapter, the data acquisition protocol is described and details on numerical methods and software used to extract the subject-specific breast geometry are described. Next, the different components of the finite element mesh are presented and the mesh quality is assessed using shape parameters.  Then, the assumptions on boundary conditions and materials models are explained. Finally, the model optimization process is detailed and results on subject-specific parameters and breast reference configuration are presented.   
\clearpage
\section{Geometry extraction}\label{section:geometryextraction}

Geometry extraction is the first step in FE analysis, and it consists of obtaining the 3D surfaces of the
breast. We use MR images to obtain the subject-specific breast volumes and the surrounding soft tissues distribution. Prior to surface extraction, the MRI volume is segmented and mapped to a single reference coordinate system. The next section describes the image acquisition protocol and the numerical method used to generate the 3D subject-specific breast geometry.

\subsection{Data acquisition}\label{subsection:imageaquisition}

 The images were acquired using a Siemens 3T scanner with T2 weighted image sequences. The in-plane image resolution was $0.5$x$0.5$ mm, and the slice thickness was 0.6 mm. During the acquisition, the contact between the breasts and the contours of the  relatively narrow MRI scanner tunnel, or with the patient body (arms or thorax), was minimized. Three different positioning configurations were considered: prone, supine and supine titled (\~ 45 deg). These positions were chosen to assess the largest possible deformations
\begin{figure}[H]
\centering
\includegraphics[width=0.9\textwidth,keepaspectratio]{figures/patientData.png} 
\caption{MR images in three breast configurations: first line - volunteer 1; second line - volunteer 2}\label{fig:patientdata}
\end{figure}

Two volunteers agreed to participate in a pilot study approved by an ethical committee (MammoBio MAP-VS pilot study). The volunteers are 59 and 58 years old and have A-cup (volunteer 1) and F-cup (volunteer 2) breast size respectively. 
 

Both volunteers were also asked to provide the compression force and breast thickness as measured during their most recent mammography. Such data are summarized in Table \ref{tab:forceandthichnessdata}.
\begin{table}[H]
\centering
\begin{tabular}{c|c|c||c|c|}
\cline{2-5}
&\multicolumn{2}{c||}{Subject 1}&\multicolumn{2}{c|}{Subject 2}\\
\cline{2-5}
& Right breast & Left breast & Right breast & Left breast\\
\cline{2-5}
\hline
\multicolumn{1}{|c||}{Force (N)}  & 21.9 &40.9 &94.8 & 56.6 \\
\hline
\multicolumn{1}{|c||}{ Breast thickness (mm)} & 47 & 42 & 50 & 49 \\
\hline

\end{tabular}
\caption{Compression force and breast thickness for both subjects for a cranio-caudal mammogram}\label{tab:forceandthichnessdata}
\end{table}

The biomechanical breast model that we propose was developed and evaluated using the MRI volumes of the first volunteer only. However, the breast deformation under compression simulations, detailed later in this manuscript,  were performed for the both breast volumes. 
\subsection{Image segmentation}%\label{subsection:imagesegmentation}

A semi-automated active contour method proposed by ITK-Snap software was used to segment the pectoral muscle and the breast tissues from MR images \citep{yushkevich_user_2006}. To segment a tissue, the image is devided in several region of interest (ROI, see Figure \ref{fig:breasttissuessegmentation}.a). For each ROI the segmentation of one tissue takes place in 3 steps (Figure \ref{fig:breasttissuessegmentation}):
\begin{enumerate}
\item First, the random forest algorithm \citep{ho_random_1995} was used to compute the probability of each pixel to belong or not to the segmented tissue. Figure \ref{fig:breasttissuessegmentation}.c shows the synthetic volume corresponding to the previously selected ROI (Figure \ref{fig:breasttissuessegmentation}.b). In the new defined volume, the pixels were classified into background (blue) and breast tissue (gray) with a given probability (color intensity). The training data set for the random forest algorithm was manually selected by the user. It includes state and space characteristics such as voxel grey intensity, voxel's neighbors gray intensity (with variable radius of neighboring) and voxel position (x, y, z) . 

\item Then, spherical regions with variable radius , also names seed points, are placed manually into the new synthetic volume. The seed points mark all the connected components belonging to the segmented tissue (Figure \ref{fig:breasttissuessegmentation}.d.1).

\item Finally, the seed point contours evolve in the 3D space with a speed and direction derived from the pixel values in the synthetic volume (Figure \ref{fig:breasttissuessegmentation}.d.2-4). In Figure \ref{fig:breasttissuessegmentation}.c, the value of a pixel belonging to the background range between $p_{val} \in [-1,0]$. Contrariwise, the value of a pixels belonging to the breast tissues range between $p_{val} \in [0,1]$. Thus the point seeds contours expend over the positive regions and and shrink over the negative ones. The propagation speed is proportional to the pixel intensity $\vert p_{val} \vert$. The process is stopped by the user then the growing region covers up the entire relevant volume.  
\end{enumerate}

 
 \begin{figure}[h]
\centering
\includegraphics[width=1\textwidth,keepaspectratio]{figures/tissues_segmentation.png} 
\caption{Breast tissues segmentation on the breast MRI of the second volunteer. Prone breast configuration.} \label{fig:breasttissuessegmentation}
\end{figure}

The breast and pectoral muscle components are computed by merging the set of segmented ROIs. After segmentation, an additional manual correction was performed to refine the boundaries of each component. Simple erosion and dilatation operations were applied on segmented volumes in order to obtain smoother surfaces. Then, to avoid tissues overlapping at juncture border between the pectoral muscle and breast, binary operations were used. First the the two segmented volumes were merged together to create a single volume. Then, from the full component the pectoral muscle volume was subtracted to redefine the breast volume without overlapping.
 
The process was repeated for both volunteers and for each breast configuration: supine, prone and supine tilted. The extracted surfaces in supine and prone configurations are used for model definition and optimization,  the last surface, in supine tilted configuration, is used for model evaluation. 

\subsection{Image registration}\label{subsection:image registration}

During the imaging acquisition process, the subject was moved in and outside the MRI scanner tunnel. Therefore, the breast dose not only undergo an elastic deformation, but also a rigid transformation. Prior to image acquisition, four landmarks were fixed on the chest wall in order to make image registration possible.  The landmarks were placed on sternum and inframammary fold lines; these regions are indeed known to be rich in fibrous ligaments limiting the soft tissue elastic deformation.  To assess the body position changes, a rigid transform is computed by minimizing the Euclidian distance between the two configurations of the four points defined by the four landmarks. The transformation is estimated using the iterative closest point (ICP) algorithm proposed by ITK library.

However, due to small local deformations of skin, the computed rigid transformation is not accurate enough. Therefore, a second registration step was performed by aligning the bone structures of the thoracic cage from prone and supine tilted positions to the supine one. The muscular tissues mask previously segmented were used in order to remove non-breast soft tissue. The image registration was implemented using a gradient descent based algorithm minimizing the image cross correlation (ITK library).

Figure \ref{fig:patientdataregistered} shows overlapping  prone-supine and supine tilted-supine breast images of the first volunteer in the transversal plane after registration. The registration error was assessed by computing the distance from the muscle's anterior surface in prone and supine tilted configurations to the muscle's anterior surface in supine configuration (Figure \ref{fig:volume_registration_error}). The chest wall line of both volumes was well aligned within an error of $ 5 \ mm$. Higher differences may be observed because of elastic thoracic cage deformations due to hand repositioning or body-mass force repartition. A maximal error of $16.25 \ mm$  and $14.9\ mm$ is observed over the region of the right arm in prone configuration and supine tilted configuration respectively, however these regions are out of the are of interest.  

\begin{figure}[!h]
\centering
\includegraphics[width=0.7\textwidth,keepaspectratio]{figures/patientDataRegisteredSubject1.png} 
\caption{Registered MRI images for the first subject, 1) prone configuration versus supine; 2) supine tilted versus supine. Yellow frame - supine configuration; Green frame - prone configuration; Red frame - supine tilted configuaration.}\label{fig:patientdataregistered}
\end{figure}

\begin{figure}[!h]
\centering
\includegraphics[width=1\textwidth,keepaspectratio]{figures/volume_registration_error.png} 
\caption{Registration accuracy measured over the anterior muscle surface.}\label{fig:volume_registration_error}
\end{figure}

As described it will be described next sections, the breast model calibration and evaluation is a time consuming procedure. Therefore, we were able to perform the breast model calibration and evaluation based on the geometry and mechanical properties of the first subject only. Therefore, the image registration of the second volunteer was not performed.

In a multi-loading gravity finite element simulation, the gravity force is applied to the whole model as a body force. The gravity force orientation can be broken down into three components of the Cartesian coordinate system labeled as X, Y, and Z directions (Figure \ref{fig:xyz_axis_directions}). The supine configuration was chosen as a reference state, therefore the gravity loading direction was set in that configuration to be oriented on the inverse direction of the Y axis (posterio-anterior direction, Figure \ref{fig:xyz_axis_directions}): $\gamma_s = (0,-1,0)$.   The gravity loading direction for the two other positions are given by the rigid transformation computed by image registration: $\gamma_p = (0.037, 0.985, -0.165)$ direction vector for gravity in prone position and $\gamma_{st} = (-0.744 , -0.667, 0.023)$ direction vector for supine tilted position. These direction vectors characterize the breast configurations of the first volunteer only.

\begin{figure}[!ht]
\centering
\includegraphics[width=0.5\textwidth,keepaspectratio]{figures/xyz_axis_directions.png} 
\caption{Anatomical planes and nominal Cartesian axis directions.}\label{fig:xyz_axis_directions}
\end{figure}

\subsection{Subject-specific 3D geometry}\label{subsection:patientspecificgeometry}

The breast subject-specific geometry was created based on the MR images in supine configuration. Following image segmentation (Figure \ref{fig:3dgeometries}.b), the outer shape of labeled regions are subsequently discretized by triangular elements.  A semi-automatic Skin Surface module proposed by SpaceClaim Direct Modeler was used to convert the mesh surfaces to non-uniform rational basis spline (NURBS) surfaces (Figure \ref{fig:3dgeometries}.c). 

\begin{figure}[!h]
\centering
\includegraphics[width=\textwidth,keepaspectratio]{figures/3dgeometries.png} 
\caption{3D geometries generation. a) MR images; b) segmented images; c) corresponding 3D geometries.} \label{fig:3dgeometries}
\end{figure}

The NURBS are averaging curves between points, therefore they are smoother and easier to use in mechanical applications. Figure \ref{fig:nurbsVSsurfaceMeshError} shows the distance from the mesh surface to the NURBS surface for the breast and muscle geometries of the first volunteer in supine position. The NURBS surface fits nearly all over the initial geometry within a tolerance of $0.5\ mm$. In areas with small curvature angles, the distance between the two surfaces increase up to $1.6\ mm$ and $3.06\ mm$ for breast and muscle geometries respectively.
 
\begin{figure}[!h]
\centering
\includegraphics[width=\textwidth,keepaspectratio]{figures/nurbsVSsurfaceMeshError.jpg} 
\caption{The distance between the mesh surface and the corresponding estimated NURBS surface. a) breast volume; b) muscle volume.} \label{fig:nurbsVSsurfaceMeshError}
\end{figure}


\section{ Finite Elements Mesh}\label{section:myFEM}

After computing the NURBS surfaces, the internal spatial information needs to be encoded using a volumetric mesh. The optimal elements type or mesh density for such a simulation is still an open question and topic of debate. The use of hexahedral elements is usually assumed to result in a more accurate solution, especially when expecting high strain/stress gradients. However, in the literature, because of the large computational time, such meshes are used mostly with a reduced number of elements \citep{ruiter_model_based_2006,gamage_modelling_2012}. Tetrahedral elements are widely used due to their geometrical flexibility and because they provide a good trade-off between computation time and estimated displacement accuracy \citep{han_nonlinear_2014,palomar_finite_2008,griesenauer_breast_2017}.   

In our case, an iterative optimization process is being considered to estimate the tissue's constitutive parameters. Therefore, to reduce the computation time, only linear tetrahedral elements were used. The first order elements are known to bear volumetric locking problems when used to model large strain for quasi-incompressible materials \citep{fung_classical_2017}. When volumetric locking occurs, the displacements calculated by the finite element method are orders of magnitude smaller than they should be. It has been shown that a linear element with a mixed U-P formulation can avoid these problems \citep{rohan_finite_2014}. Therefore, in our work, the geometries are meshed using the tetrahedral element solid285 (ANSYS Mechanical) which provides a mixed U-P formulation option. 

 On the other hand, the mesh density has also an impact on model accuracy. A finer mesh result in a more accurate and stable solution, but also increase the computational time. To our knowledge, no studies have determined the optimal resolution of the volumetric mesh for simulating breast tissues deformations. To determinate the appropriate mesh size, a mesh convergence study was performed. The details can be found in Appendix \ref{appendice:meshconvergence}.  According to these results the optimal element’s size ranges between 7 and 10mm. Thus, the mesh that was chosen for the first volunteer consists in 18453 tetrahedral elements with 9625 elements assigned to the pectoral muscle and the thoracic cage and 8828 elements assigned to breast tissues (Figure \ref{fig:meshcomponents}).
 
 \begin{figure}[!h]
\centering
\includegraphics[width=1\textwidth,keepaspectratio]{figures/mesh3components.png} 
\caption{ Finite element mesh components. The tissues components are cropped for visualization purposes. }\label{fig:meshcomponents}
\end{figure}
 
 The mesh quality was measured using three criteria: element skewness, aspect ratio and maximal corner angle. The Figure \ref{fig:meshquality} shows the values ranges distribution for these shape parameters. The element's aspect ratio and maximal corner angle range between the nominal limits defining a good mesh quality (Section \ref{section:lagrangianmesh}). There is a small number of elements with a skewness lager than the maximal theoretical quality limit ( 0.75 ) , however there are no degenerated elements (skewness = 1).  

\begin{figure}[!h]
\centering
\includegraphics[width=1\textwidth,keepaspectratio]{figures/meshquality.jpg} 
\caption{ Finite elements mesh quality.}\label{fig:meshquality}
\end{figure}




The breast skin layer is added a posteriori   as a $2mm$ thick single layer of shell elements (1980 elements). Shell elements and the underlying solid elements are sharing the same nodes (Figure \ref{fig:meshquality}.a).



\section{Breast reference configuration}\label{section:myStressFree}
To estimate the reference configuration of the breast (\textit{stress-free} configuration), an adapted prediction-correction iterative approach was implemented \citep{eiben_breast_2014} using prone and supine image data sets. The overall iterative process is presented in Figure \ref{fig:myfixedpointalgo}. The first estimate of stress-free breast configuration is computed by applying an inverse gravity on the supine geometry. Then, at each iteration, the estimated stress-free configuration is used to simulate breast deformation due to gravity in a prone position. The differences between the result of this simulation and the real shape of the breast in prone position is quantified by computing the Euclidian distances $d_j$ for each \textit{active node} defined at the breast external surface. These distances are then used in the next iteration of the process to simulate imposed displacements (Dirichlet condition) to the active nodes j in the stress-free configuration. To reduce the mesh deformation, and thus to limit any element distortion, the displacements are only partially imposed using a multiplicative regularization factor $ (\alpha\ <\ 1)$ \nomenclature{$\alpha$}{Multiplicative regularization factor}. The process repeats as long as the new transformation improves the similarity between the two geometries in prone configuration by more than $1\ mm$ on average, $\vert D_{i+1}-D_i \vert > 1\ mm$. The similarity between the estimated and measured configurations at iteration $i$ is given by the mean Euclidean distance $D_i$ over the active nodes $j$.                                                              

\begin{figure}[!h]
\centering
\includegraphics[width=0.85\textwidth,keepaspectratio]{figures/stress_free_config_algo.jpg} 
\caption{Fixed point type iterative algorithm for stress-free geometry approximation. $D_i$ - mean node to node distance over the active nodes at iteration $i$, $G$ - gravity force}\label{fig:myfixedpointalgo}
\end{figure}

During the fitting process, the active nodes are chosen to be the ones corresponding only to the breast surface. The skin nodes belonging to the  arms and the lateral thoracic areas are excluded  in order to neglect as much as possible the error due to rigid body changes.

To compute the distance $d_j$ between the node position in the estimated and the measured breast configurations\nomenclature{$d_j$}{Distance between the node position in estimated and measured breast prone configurations}, the  positions of the active surface nodes on prone configuration have to be known. Thus, an additional mesh registration step was performed at each iteration: the active nodes were morphed into prone configuration using the elastic deformation method proposed by Bucki M. \citep{bucki_fast_2010}. The method estimates a C1-diffeomorphic, non-folding and one-to-one transformation to register a source point cloud onto a target data set, which can either be a point cloud or a surface mesh.  The set of input source points is initially embedded in a deformable virtual hexahedral elastic grid. Then, an iterative registration technique is performed. At each iteration the grid is deformed such that the distance between the target and the source nodes is minimized. The use of the grid allows to speed-up the registration process. The distances between the grid vertices and target nodes are computed only once, prior to registration. The source points location is then computed by interpolation between the closest grid vertices. To increase the registration accuracy, the regular grid is progressively refined by subdividing each cell into eight smaller ones. Seven refinement steps were performed resulting in a minimal cell size of about $2\ mm$. 

\section{Boundary conditions}\label{section:myBoundayconditions}
To provide a rigid support for the muscle mesh component, zero displacement conditions are imposed to its posterior face, assumed to be attached to the rib cage (figure \ref{fig:meshboundaries}). Then, the interface between the breast mesh and muscle mesh is modeled using contact mechanics. The muscle is stiffer than the adipose tissues, thus its anterior face represents the target surface and the posterior breast face represents the contact surface (see Section \ref{section:contactmechanics} for a reminder of these target and source surfaces). 

\begin{figure}[!h]
\centering
\includegraphics[width=0.9\textwidth,keepaspectratio]{figures/mesh_parts_2.png} 
\caption{Finite elements mesh boundaries}\label{fig:meshboundaries}
\end{figure}

Previous works have shown that modeling breast deformations from prone to supine configurations requires taking into account breast tissues sliding over the chest wall \citep{carter_application_2012,han_nonlinear_2014}.  Therefore, the juncture surface is modeled as a no-separation contact with a frictional behavior proposed by the ANSYS Contact Technologies (see Section \ref{subsection:surfaceinteractionmodels}). The penalty algorithm is used with a meticulous control of contact normal and opening stiffness parameters. At this stage, stiffness parameters do not have a physical meaning and have to be identified by \textit{trial and error} methods. Since these parameters are extremely sensitive to the stiffness of the underlying elements and to the direction of the local deformation, new values have to be identified for each new simulation case.    


To study the impact of the friction coefficient ($\mu_f$) on tissues sliding, several simulations have been performed at different values of $\mu_f$. We found that, with the Coulomb friction law, even for a high value of $\mu_f$, too much sliding is allowed when estimating the prone breast configuration. At the contact surface, because of excessive sliding, the tissue accumulation in the region of the sternum line results in a sinuous surface (Figure \ref{fig:overslidingProblem}); thus, the finite element mesh  undergoes important distortions and the solution is compromised. Therefore, different strategies based on anatomical breast structures were investigated to limit the amount of sliding and to overcome solution instabilities (Appendix \ref{appendix:contactSurface}). However, it seams that a small amount of friction improves the solution convergence capabilities \citep{ansys_contact_2017}; therefore the friction coefficient was kept to $\mu_f = 0.1$. 
\begin{figure}[!h]
\centering
\includegraphics[width=0.7\textwidth,keepaspectratio]{figures/overslidingProblem.jpg} 
\caption{Tissues acumulation on the sternum line with excessive sliding}\label{fig:overslidingProblem}
\end{figure}
 
The breast soft tissue is firmly attached to the deep fascia via suspensory ligaments but can move freely over the pectoralis muscle  \citep{mugea2014aesthetic,clemente2011anatomy}. Therefore, the strategy chosen to control the amount of tissues sliding relies on ligamentous breast structures described in Section \ref{subsection:internalstructures}. As far as  breast support matrix is concerned, only the largest structures were modeled (i.e. fascias and suspensory ligaments).  The superficial layer of the superficial fascia was integrated in the skin layer, assuming a higher material stiffness. In addition, a new layer of 0.1mm thick shell elements was added at the juncture surface between muscle and breast tissue to model the deep layer of the superficial fascia. The shell and the underlying breast elements were imposed to share the same nodes. Since the deep fascia and muscle tissues are supposed to present similar elastic properties, the deep fascia was not explicitly modeled. In addition, two ligamentous structures (inframammary ligament and deep medial ligament) were modeled using Ansys link type elements connecting the node of the breast posterior surface to anterior muscle surface (Figure \ref{fig:mesh_components_BC}).  


Several additional Dirichlet conditions were set on the mesh boundaries: the superior and inferior ends of the deep fascia layer are constrained in Z direction; the superior and inferior ends of the skin layer are constrained in Y direction. For left and right lateral breast boundaries (Figure \ref{fig:meshboundaries}), Dirichlet conditions are too strong and preclude breast tissue to slide laterally. Therefore, in these regions, new ligamentous structures are included using link elements with a cable-like behavior (Figure \ref{fig:mesh_components_BC}).



\begin{figure}[!h]
\centering
\includegraphics[width=0.9\textwidth,keepaspectratio]{figures/mesh_components.png} 
\caption{Components of the finite element mesh.}\label{fig:mesh_components_BC}
\end{figure}

 The deep layer of the superficial fascia is much stiffer than the underlying adipose tissues. Due to imposed boundary conditions, the amount of sliding depends on fascia's elasticity. The suspensory ligaments define regions where the breast sliding is minimal regardless of the applied deformations. These additional stiff structures reduce the tissues sliding and improve the solution convergence capability. 

\section{Material constitutive models}
\label{section:myConstitutivModels}

Our final model consists of 6 types of tissues, wherein 4 tissues (glandular, fatty, muscle and skin) are well described and regularly used in biomechanical modeling, and 2 of them (fascia and suspension ligaments) are with limited use and poorly described in the literature. A large range of values of the constitutive parameters are available for each tissue. However because of an inconsistent interindividual variability, subject-specific parameters had to be identified.  


 Here, all materials except to the suspensory ligaments were modeled using the Neo-Hookean strain energy functions. Breast suspensory ligaments were assumed to undergo only small deformations, thus they were considered as linear materials.  Subject-specific mechanical tissue properties were computed using an optimization process based on a multi-loading gravity simulation procedure (Figure \ref{fig:optimizationalgo}). First, for a given set of parameters $(\lambda_{glandular}$\nomenclature{$\lambda_{xx}$}{Youngs's moduli of the material $xx$}, $\lambda_{adipose}$, $ \lambda_{muscle}$,  $\lambda_{skin}$, $\lambda_{fascia}$, $\lambda_{ligam}$, $\nu$ ), the stress-free configuration was estimated by minimizing the difference between the simulated and the measured breast geometry in prone configuration. Then, from the new estimated stress-free geometry, the supine breast configuration was derived. The estimated supine geometry was compared to the measured one using modified Hausdorff distance (see Appendix \ref{appendice:distancemeasures} for distance definition), representing the estimation error.  To avoid including the geometry dissimilarity due to arm position, the modified Hausdorff distance was computed only on breast skin surface.  
The process implies multiple simulations based on imposed node displacements; therefore the FE mesh can be significantly altered (with convergence issues) before reaching an optimal stress-free geometry. Mainly for that reason, we chose to perform an exhaustive manual research (rather than an automatic one) of the optimal set of constitutive parameters. 


\begin{figure}[!h]
\centering
\includegraphics[width=0.9\textwidth,keepaspectratio]{figures/optimizationMaterialParameters.png} 
\caption{Process to estimate optimal material parameters}\label{fig:optimizationalgo}
\end{figure}
 
 An optimization process including finite element simulations with 6 parameters results is a complex and time-consuming problem. The model simplification was then performed in two steps. First, the parameters which variations have limited effects on simulation results were identified and set to an optimal fixed value. Next, for parameters which variations have a high impact on simulation results, a sensitivity study was performed to redefine the search intervals and interval's discretization step.
 
 \subsection{Model simplification}

The breast tissues are mainly composed of water; a usual assumption is to consider them as nearly incompressible materials\citep{fung_biomechanics_2013}. However,
previous works proposed a Poisson's ratio value ranging between $\nu = 0.3$ \citep{hopp_automatic_2013} and $\nu = 0.5$ \citep{gamage_modelling_2012}. In a multi-loading gravity simulation, the breast volume is nearly constant, thus, the influence of Poisson's ratio on nodes displacements was studied only for values ranging between $\nu = [0.45 , 0.495]$ (Figure \ref{fig:poissonRatio}). 

\begin{figure}[!h]
\centering
\includegraphics[width=\textwidth,keepaspectratio]{figures/poissonRatio.jpg} 
\caption{Estimation of the optimal material parameters}\label{fig:poissonRatio}
\end{figure}

The simulations were performed by applying the gravity force in posterio-anterior direction on breast geometry in supine configuration. We present on Figure \ref{fig:poissonRatio}, the node displacement variation and the change rate of node displacement as function of the tissue Poisson ratio. On the left-hand side, the mean and the maximal displacements of the skin nodes are given for each values of $\nu$. On the right-hand side, the maximal difference of nodal displacements between two consecutive simulations (change rate) and the maximal difference of node displacements between the actual and the less deformed geometry (cumulative change rate) are plotted. The change rate is computed within the assumption that the maximal displacement over the simulations set represents $100\ \%$ change rate. Non-significant variations are observed on the mean and maximal displacements of skin nodes, thus a constant value of $\nu = 0.49$ was chosen.


The pectoral muscle together with the thoracic cage are the breast tissues support. The deformation of the muscle under gravity loading is neglected. Therefore, its Young's modulus was not included on the parametric study and was chosen so that only minimal deformations occur ($\lambda_{muscle}=10kPa$).

The ligamentous breast structures are added with a cable-like behavior to reduce tissues sliding. Their Young's modulus is also not included in the optimization process and is set sufficiently high ($\lambda_{ligam}=500kPa$) to preclude their elastic deformation. 

The adipose and glandular tissues are known to be extremely soft and to undergo large deformations under gravity loading. Calvo-Gallego J.L. \citep{calvo_polynomial_2015} proposed a uniform polynomial material model for the mixture of adipose and glandular tissues. The authors have also shown that the breast outer shape deformation does not depend on glandular distribution but is highly dependent on its volumetric ratio. In this work,  it was chosen to model the glandular and fatty tissues as a single homogeneous material with an equivalent Young's modulus $\lambda_{breast }$. The mechanical properties of the equivalent breast tissue are in direct relation with glandular and adipose volumes ratios. Because the left and right breasts may have different glandularities, two different parameters are considered ($\lambda_{breast}^l$ \nomenclature{$\lambda_{xx}^{l/r}$}{Young's modulus of the material xx for the left/right breast}, $\lambda_{breast}^r$), one for each laterality.

Breast skin and superficial fascia are an essential part of the breast support matrix. The two layers are much stiffer than breast tissues and are the structures governing the amount of deformations. Their elastic behavior was included on the optimization process.

 Based on existing publications, an interval of possible values are given in Table \ref{table:minandmaxelasticmodulus} for each parameter included in the optimization process $(\lambda_{breast}^l$, $\lambda_{breast}^r$, $\lambda_{skin}$, $\lambda_{fascia})$. To characterize model sensitivity to parameters' variations, a set of simulations were performed. The defined intervals for each type of tissues were discretized by steps of 10\% and at each step the skin nodes displacement were computed. Results of the corresponding simulations are shown on Figure  \ref{fig:materialPropDiscretization}. As previously, the first column represents the variation of mean and maximal displacements of skin nodes in function of the elastic parameter of each material; the second column represents the change rate and the cumulative change rate of skin nodes displacements.


\begin{table}[!h]
\centering
\begin{tabular}{|c||c|c|c||c|c|c|}
\hline
&\multicolumn{3}{|c||}{Search intervals}& \multicolumn{3}{c|}{Search intervals}\\
&\multicolumn{3}{|c||}{ from bibliographic data}& \multicolumn{3}{c|}{ after model simplification}\\
\hline
\hline
& Breast & Skin & Fascia & Breast & Skin & Fascia \\
\hline
Min (kPa)  & 0.3 & 7.4 & 100 & 0.3 & 2 & 50\\
\hline
Max (kPa) & 6 & 58.4 & 5000& 4& 20 &250 \\
\hline
\end{tabular}
\caption{Minimal and maximal value (in kPa) for Young's moduli.}
\label{table:minandmaxelasticmodulus}
\end{table}

\begin{figure}[!h]
\centering
\includegraphics[width=\textwidth,keepaspectratio]{figures/materialPropDiscretization.png} 
\caption{First column: relation between maximal and mean nodes displacement and the equivalent Young's moduli variation for different tissues. Second column: rate and cumulative change rate of skin node displacement as function of quivalent Young's modulus}\label{fig:materialPropDiscretization}
\end{figure}

The figure shows that the model is very sensitive to the variation of Young’s modulus of breast tissue, skin and fascia (Figure \ref{fig:materialPropDiscretization}). However, beyond some values, the materials become too stiff and do not change significantly the breast deformations under gravity loadings.  Therefore, the search intervals for breast tissue and skin Young's moduli were reduced so that a larger value impacts the cumulative change rate less than 20\% (max displacement less than $\sim 5mm$). As the fascia stiffness governs the lateral displacement and shows a smaller variation over the search interval, a threshold of 10 \% ($\sim 2.5mm)$ was chosen. The obtained search intervals are summarized in Table \ref{table:minandmaxelasticmodulus} 



\subsection{Estimation of the optimal constitutive parameters}

The model optimization is a tough and time consuming process. It was extremely difficult to obtain the solution convergence when combining the tissues large deformations with the sliding contact conditions. Because of the large computation time, the reference breast configuration and the optimal constitutive parameters were computed only for the first volunteer. The model optimization of the second volunteer is considered for future work.

To estimate the constitutive parameters describing the breast mechanics of the first volunteer, the new intervals defined by the above sensitivity analysis were discretized by steps of $0.1 kPa$, $1kPa$ and $40 kPa$ for breast, skin and fascia tissues respectively. The discretization step was chosen such that the change rate between two consecutive simulations is less than 10\%. The previously described multi-loading gravity process was performed for each set of parameters and the corresponding model error distribution is shown in  Figure \ref{fig:optimizationresults}. The contour lines are estimated by linear interpolation between two consecutive simulations.

\begin{figure}[!h]
\centering
\includegraphics[width=0.9\textwidth,keepaspectratio]{figures/optimizationMaterialParameters2.png} 
\caption{Hausdorff distance on the skin surface over the constitutive parameters space}\label{fig:optimizationresults}
\end{figure}

It was found that the value of the Young's modulus of the breast tissues is lower than the ones reported in the bibliography. Therefore more simulations were done outside the defined intervals. However, for very low values, below 0.2 kPa, 2 kPa and 80 kPa for breast, skin and fascia's Young's modulus respectively, the tissues deformation is too large and the finite element mesh becomes degenerated at the first step of the multi-loading simulation. For values above 1 kPa, 5 kPa and 160 kPa, tissues deformation is too small compared to the ones observed from the MR images and the simulations using such values were excluded. All other missing values correspond to failed simulations due to a non-converging force, specifically in the region of the contact surface between the breast and the muscle.

 

For soft fascia ($\lambda_{fascia} = 80kPa$), the lateral displacement of breast tissues is more important than the one measured on MR images. Contrariwise, for stiff fascia material ($\lambda_{fascia}=160 kPa$) the amount of sliding is too small. To match the breast geometry in supine configuration, very low values of breast Young's modulus are required ( $\lambda_{breast}< 0.2kPa$). For such low values, the breast tissues are highly deformed, thus the finite elements undergo large distortions. Due to such errors in element formulation, the simulations giving the minimal Hausdorff distance did not succeed.   

The set of parameters providing the best match between simulated and measured supine breast configurations is ($\lambda_{breast}^r=0.3 kPa$, $\lambda_{breast}^l=0.2 kPa$, $\lambda_{skin}=4 kPa$, $\lambda_{fascia}=120 kPa$).  

Figure \ref{fig:optimizationresults} shows the differences between the measured and estimated breast geometries in prone(left) and supine(right) configurations . Each distance was computed over the skin active nodes used also for the model optimization. 

\begin{figure}[!h]
\centering
\includegraphics[width=\textwidth,keepaspectratio]{figures/optimizationresults.png} 
\caption{Difference  between estimated and measured data, in prone and supine configurations, obtained with optimal Young's moduli and stress-free geometry. }\label{fig:geometryoptimizationresults}
\end{figure}

The breast geometry is better estimated in supine configuration, with an Hausdorff distance equal to 1.72 mm. This is probably due to a better representation of the boundary conditions in supine configuration, as this configuration was used to create the initial finite element mesh. The breast geometry in prone configuration is also well estimated, with a modified Hausdorff distance equal to 2.17 mm. The maximal node to surface distance is obtained on the breast lateral parts.  Considering non uniform skin thickness or elastic properties over the breast surface, as described by Sutradhar A. and Miller M.J. \citep{sutradhar_vivo_2013}, should improve the obtained results in prone configuration. 




